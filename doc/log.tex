% vim: sw=2 ft=tex
%         File: log.tex
% Date Created: 2013 Oct 14
%  Last Change: 2013 Nov 07
%       Author: csulbj
\author{Alex J. Best}
\date{\today}
\title{Finding orders with prescribed index\\Project Log}
\documentclass[11pt,a4paper]{article}
\usepackage{amsmath, amssymb, amsfonts}
\usepackage{fullpage}
\usepackage[utf8]{inputenc}
\usepackage[T1]{fontenc}
\usepackage[english]{babel}

\newcommand{\oK}{\mathfrak{o}_\mathcal{K}}

\begin{document}
\maketitle

\marginpar{13/10/13}
Implementations exist in Magma (refs)
Finding power basis for order with specified index in a given order.

\marginpar{15/10/13}
Istv\'an Ga\'al has done a lot of work in this area, and there exists(ed) an implementation within KANT (run by Pohst).

\marginpar{16/10/13}
Should think up naive algorithm first in a general setting (no power basis etc).
Take an absolute number field $K|\mathbb{Q}$ and consider $\mathcal{O}\subset \mathcal{O}_K$ letting $I=[\mathcal{O}_K:\mathcal{O}]$ we should find a relation between $I$ and the conductor of $\mathcal{O}$.
Recall the conductor $\mathfrak{f}$ of $\mathcal{O}$ is the largest ideal of $\mathcal{O}_K$ contained in $\mathcal{O}$, so $\mathfrak{f} = \{\alpha\in\mathcal{O}_K\mid \alpha\mathcal{O}_K\subseteq \mathcal{O}\}$.
Will probably come down to computing some quotient $\mathcal{O}/\mathcal{N}$.

\marginpar{25/10/13}
We have that $\mathcal{O}_K/\mathcal{O}\cong (\mathcal{O}_K/\mathfrak{f})/(\mathcal{O}/\mathfrak{f})$ so we can see $[\mathcal{O}_K:\mathcal{O}][\mathcal{O}:\mathfrak{f}] = [\mathcal{O}_K:\mathfrak{f}]$.
As we know the index we wish to obtain we know that the norm of the conductor of the order we are looking for is divisible by the index.
This gives us some prime ideals that must divide the conductor so we start with these.
We then calculate the possible conductors and their corresponding orders (how?).

\marginpar{7/11/13}
Given $[F:\mathbb{Q}] = d$ we have $\mathcal{O}_F\subset F$ and $\mathcal{O} = \mathbb{Z} + f\mathcal{O}_F$.
Then $\mathbb{O}_F/\mathcal{O} = \mathcal{O}_F/f\mathcal{O}_f$ torsion module.
$\operatorname{norm}(f\mathcal{O}_F) = f^2 \in \mathbb{Z}$.

Can work one prime at a time (locally) then path things together for a speedup.

\begin{table}[h]
\begin{tabular}{llllllllllllllllllllllllllllllllllll}
a&b&c&d&e&f&g&h&i&j&k&l&m&n&o&p&q&r&s&t&u&v\\
$\mathbb{a}$&$\mathbb{b}$&$\mathbb{c}$&$\mathbb{d}$&$\mathbb{e}$&$\mathbb{f}$&$\mathbb{g}$&$\mathbb{h}$&$\mathbb{i}$&$\mathbb{j}$&$\mathbb{k}$&$\mathbb{l}$&$\mathbb{m}$&$\mathbb{n}$&$\mathbb{o}$&$\mathbb{p}$&$\mathbb{q}$&$\mathbb{r}$&$\mathbb{s}$&$\mathbb{t}$&$\mathbb{u}$&$\mathbb{v}$\\
w&x&y&z&1&2&3&4&5&6&7&8&9&0 \\
$\mathbb{w}$&$\mathbb{x}$&$\mathbb{y}$&$\mathbb{z}$&$\mathbb{1}$&$\mathbb{2}$&$\mathbb{3}$&$\mathbb{4}$&$\mathbb{5}$&$\mathbb{6}$&$\mathbb{7}$&$\mathbb{8}$&$\mathbb{9}$&$\mathbb{0}$ \\
\end{tabular}
\caption{\label{}Table of mathbb symbols and their intended counterparts}
\end{table}


\section*{Translation of \"{U}ber das L\"{o}sen von Einheitenund
Indexformgleichungen in algebraischen
Zahlk\"{o}rpern mit einer Anwendung auf die
Bestimmung aller ganzen Punkte einer
Mordellschen Kurve - Chapter 2
}
If $\mathcal{K}$ is an algebraic number field so has for any $I \in\mathbb{N}$ the set $\{\alpha\in\oK\mid ( \oK: \mathbb{Z}[\alpha]) = I\}$ for a result of Gy\H{o}ry [25] a finite representative system $\mathfrak{I}_\mathcal{K}(I)$ with respect to $\mathbb{Z}$-equivalence, where two integral algebraic numbers $\alpha,\beta\in \oK$ are called $\mathbb{Z}$-equivalent if $\alpha\pm\beta\in\oK$.
The calculation of $\mathfrak{I}_\mathcal{K}(I)$ is called the index form equation.
These designations stirred therefore, because an integral basis $\omega_1 = 1,\omega_2,\ldots,\omega_n$ of $\oK$ of a form $\mathfrak{I}_\mathcal{K}(t_1,\ldots,t_n) \in \mathbb{Z}[t_2,\ldots,t_n]$ with the property exists, since for all $\alpha = x_1\omega_1 +\ldots + x_n\omega_n \in \oK$ with $( \oK : \mathbb{Z}[\alpha]) <\infty$ each 
\[( \oK : \mathbb{Z}[\alpha]) = \pm I_\mathcal{K}(x_2,\ldots, x_n)\]
applies.

The shape IK called index form of K with respect to 1; ::: ; ! N .
We will in this chapter apply our method for equations on the unit to lots of index form equations.
We mainly use the classic method by which the finiteness of Gy\H{o}ry IK ( I) showed that, the calculation of the efective IK ( I) on the lots of finitely many units -
%equations in the Galois closure of K zuruckfuhrte . With our development
%of Gy } rys method to an algorithm , which is partly based on two papers
%o
%Smart 50 , 51] supports , we drew lots for the first time index form equations in number fieldn
%of degree 8 , 10 and 12 In addition, we used an idea of Niklasch 39 ] to
%Kreisteilungskorpern the Q ( 17) Q ( 19) and Q ( 23) all power wholeness bases
%to be calculated.
%The applicability of Gy } rys method is in practice restricted by the
%o
%Necessity drove through bills in the Galois closure of K to need .
%Alternative methods for lots of index form equations , which is not the ga -
%loisschen closure bedurfen exist so far only for number field of degree 4 for
%cubic number field showed Gaal and Schulte 20] , since the determination of IK ( I)
%is equivalent to the lots of a cubic Thue equation for quartic number field
%Gaal developed , and Peth } Pohst 17, 18] a method in which the calculation
%essentially on the toss of a ternary system of IK ( I) For square -
%men will zuruckgefuhrt . In addition, there are methods of Gaal, 15, 16]
%for and by Gaal and Pohst 19] special body of sixth degree .
%To further formulation , we define some notations. As usual in this
%Work is given as K = Q K () with a whole algebraic number . we
%choose d 2 N with Doc Z] . Furthermore, we denote the Galois closure of K
%with L and set m: = L: Q] . For the Galois group of L we write Q = G.
%Since lots of index form equation for n = K: Q ] 2 is trivial , is without
%Restriction n 3
%we use
%N : = f1 ; ::: ng \nonumber\\
%N2: = ffi j g j 1 i <j ng \nonumber\\
%N3: = ffi j ; kg j 1 i <j <k ng :
%The Galois group G was operating on the amounts of N , N2 and N3 vermoge
%G N! N : ( i) 7!
%i = i0 , where ((i) ) = ( i0 )\nonumber\\
%G N2 ! N2 : ( fi j g) 7! fi j g : = f i j g\nonumber\\
%G N3 ! N3 : ( fi , j , kg) 7! fi , j; kg: = f i , j; kg:
%There are
%2
%3
%: = Fw1 ; ::: ws g : = fG j 2 N 2 g\nonumber\\
%: = Fw1 ; ::: g Wt : = fG j 2 N 3 g
%the quantities of the orbit with respect to the operations of G on N2 and N3. the
%Values ​​of s and t , which exclude Lich depend on n and G, we sometimes write
%also known as s ( n , G), and t ( n , G).
%We choose for each j 2 f1 ; ::: sg a fixed j 2 j, and similarly for each k 2 !
%f1 ; ::: tg a fixed k 2 Wk . Is fi = j ; 2 kg N3 given with i < j <k ,
%we de ne (1): = fi , jg , (2) : = fj , kg, and (3): = fi ; kg. 2 to N2
%Let F be the stabilizer of the Fixkorper respect to the action of G on N 2.
%Nally we set K: = Q ((i) , ( ​​j )) for = fi , jg 2 N2 and more K: =
%Q ((i ), ( j ), ( k )) for fi = j ; 2 kg N3.
%For every 2 IK (I) and each fi = ; jg 2 N2 is the choice of d
%( i)
%( j)
%: = D ((i) ( j) ? )
%?
%an algebraic integer of F . it is
%dn ( n ? 1) I 2 disck = dn ( n ? 1)
%Y
%fi ; jg2N2
%((I) : ( j)) = 2
%(2-2 )
%Y
%fi ; jg2N2
%((I) ( j ?)) 2 2 \nonumber\\
%ij
%where we in the right product ij instead of fi , jg wrote | a read -
%simplification , which we in the following apply indexing with elements
%retained for N2 and N3. Let
%n (n ? 1) disc
%I 0 = I 2 d disc Z] K 2 N
%33
%and we choose Gj G minimal with Gj j = wj (1
%every 2 IK ( I), the equation
%I0 =
%Y
%s
%YY
%2
%ij =
%fi ; jg2N2
%j = 1 2GJ
%(
%j s ) , we obtain for
%j)
%2:
%(2-3)
%The first step in the calculation of IK ( I) will consist of a finite
%F to determine s such that for every 2 IK (I) a =
%A lot of F 1
%(a 1 ; ::: a s ) 2 A exists with
%j
%2 a j j UF
%(1 j
%s ) :
%(2-4)
%In the case of I = d = 1, that I 0 = 1 Obviously, A = f o provides (1 ; ::: 1) g of the Desired .
%If, however, I 0 > 1 , then the determination of A is by far more difficult to -
%A times with respect to the second or following step for calculating the k ( I)
%mogli waxed should contain a few entries that are not added to a solution of IK ( I)
%gema (rther
%( pe1j
%1
%2GJ
%perj2
%r )
%(
%P1 = 1e)
%( e)
%pr r\nonumber\\
%where for each i 2 f1 ; ::: rg is the exponent i (s ) is a linear function in E
%i (s) =
%r s
%XX
%= 1 = 1
%s
%i
%is the coe Zieten i 2 N 0 are defined by
%Y
%(p) 2 = 1 p1
%2G
%Let
%0
%B
%L : = B
%@
%pr r
%r , 1
%(1
%111
%:::
%1R1
%:::
%11s
%:::
%1rs
%r11
%:::
%rr1
%:::
%r1s
%:::
%rrs
%..
%.
%..
%.
%..
%.
%..
%.
%s ) :
%1
%C.\nonumber\\
%C
%A
%(2-6)
%Accordingly, as the amount
%L : = fx 2 N rs
%0
%j (1 ; ::: r) t = L xg
%(2-7)
%finally , and it is e.
%remark 2.1
%The so- constructed set A contains usually a large number of elements ,
%what not to solutions of IK ( I) gema (2-4) belong . Some of these elements
%can be screened using the following two criteria :
%1 Let I = 1, and is further 2 IK ( I) arbitrarily . Then Z] Z] applies . for
%each = fi j g 2 N2 thus shares ( i) ? ( j) the di erence ( i) ? ( j) in oF.
%Thus, for a mu (a 1 ; ::: as) 2 A each aj is a divisor of d oF be
%(1 j s ) . PDR is pd1 the prime ideal of oL d so for valid
%1
%r
%each (x11 ; ::: r1 ; ::: x1s ; ::: xrs ) t 2 L, which with a solution of IK ( I)
%corresponds , therefore x j d (1
%r , 1 j s ) .
%2 Let K be normal with G abelian , and were also 2 IK (I) and fi , jg 2 N2
%arbitrarily set . For each 2 G is then due
%( d ((i) : ( j ))) = d ( (()) ( i) : ( ()) (j))
%the di erence ( i) ? ( j) a divisor of (d ((i) : ( j ))) in oK .
%For a (a 1 ; ::: a s ) 2 A so mu
%((I) : ( j)) ( a)
%s, fi , j = g )
%(1
%( i) ? ( j)
%always be an algebraic integer of ok .
%example 2.2
%Let K = Q () , where zero of t5 ? 10 t3 + t2 + 10 t 5 + 1 Then K is normal
%with Galois C (5) , and K 1 has class number An integral basis of oK
%is given by
%2
%3
%4
%! 1 = 1; ! = 2 , ! 3 = 2 , ! 4 = 3; ! 5 = 2 + 2 + 6 + 7 + 3 \nonumber\\
%and by ( ok : Z]) = 7, we can choose d = 7 .
%We want iK 2 N minimal with IK ( iK ) = 6 , and calculate the representative system
%IK ( iK ) calculated explicitly . Since K is not the real maximum Teilkorpe2 , 3, 5, 7 in each pairwise different prime ideals in out ok :
%Ok 2 = p 1 \nonumber\\
%It is particularly
%3 oK
%= P2 \nonumber\\
%5 ok
%= P5 \nonumber\\
%3
%7 ok
%P5P6 = p4 p7 p8 :
%dn ( ? n 1) Disck o = 718 o = p18 p18 p18 p18p18 :
%K
%4 5 6 7 8
%Z disc ] K
%Furthermore, we note
%2
%= Fff1 , 2g , f2 , 3g , f3 , 4g , f4 , 5g , f1; 5gg ; ff1 ; 3g , f2 , 4g , f3 , 5g , f1 , 4g , f2 ; 5ggg \nonumber\\
%so is G = G1 = G2.
%35
%Case 1 I = 2
%It is k ( 2) = , as the linear system
%0 2 1 0 10
%B 18 C B 0
%B C B
%B 18 C B 0
%B C = B
%B C B
%B 18 C B 0
%B C B
%B 18 C B 0
%@ A @
%0
%2
%2
%2
%2
%0 2
%18
%0
%2
%2
%2
%2
%2
%0
%2
%2
%2
%2
%2
%0
%2
%2
%2
%2
%2
%0
%2
%2
%2
%2
%2
%10
%0
%0
%0
%0
%0
%0
%2
%2
%2
%2
%2
%0
%2
%2
%2
%2
%2
%0
%2
%2
%2
%2
%2
%0
%2
%2
%2
%2
%2
%0
%2
%2
%2
%2
%2
%1
%C
%C
%C
%C x
%C
%C
%C
%C
%A
%from (2-7 ) has no solution in N 12th
%0
%Case 2 I = 3
%It is k ( 3) = ; , since the same system of equations as in the case I = 2.
%Case 3 I = 4, 6
%It is k ( 4) = ; = IK (6 ) how easily the basis of the equations of
%Fall I = 2 and I = 3 concludes .
%4.Fall I = 5
%We obtain the linear equation system
%0 10 1 0 10
%B 18 C B 0
%B C B
%B 18 C B 0
%B C = B
%B C B
%B 18 C B 0
%B C B
%B 18 C B 0
%@ A @
%0
%2
%2
%2
%2
%0 2
%18
%0
%2
%2
%2
%2
%2
%0
%2
%2
%2
%2
%2
%0
%2
%2
%2
%2
%2
%0
%2
%2
%2
%2
%2
%10
%0
%0
%0
%0
%0
%0
%2
%2
%2
%2
%2
%0
%2
%2
%2
%2
%2
%0
%2
%2
%2
%2
%2
%0
%2
%2
%2
%2
%2
%0
%2
%2
%2
%2
%2
%1
%C
%C
%C
%C x
%C
%C
%C
%C
%A
%with slogans in 97240 N 12th From this we can 93236 vermoge many of the second
%0
%Sort of criterion 2.1 . For I = 5 A thus contains 4004 elements .
%5.Fall I = 7
%We obtain the linear equation system
%0 1 0
%B 20 C 2 B
%B 20 C 2 B
%B C = B
%B 20 C 2 B
%B C B
%B 20 C 2 B
%@ A @
%2
%2
%2
%2
%2 2
%20
%2
%2
%2
%2
%2
%2
%2
%2
%2
%2
%2
%2
%2
%2
%2
%2
%2
%2
%2
%2
%2
%2
%2
%2
%2
%2
%2
%2
%2
%2
%2
%2
%2
%2
%2
%2
%2
%2
%2
%2
%1
%C
%C
%C x
%C
%C
%C
%A
%with slogans in 92378 N 10th From this we can 87373 vermoge many of the second
%0
%Sort of criterion 2.1 . For I = 7 A thus contains 5005 elements .
%We will continue this example later .
%For the remainder of this chapter is a = (a 1 ; ::: as) 2 A arbitrary but fixed identifier coded
%ben . Let
%IK ( I, a) : = f
%2 IK ( a) j is equivalent to a vermoge (2-4) g
%and more
%a
%j
%: = (A j)
%( 2 G , 1 j s )\nonumber\\
%36
%where a
%each
%CHAPTER 2 INDEX FORM EQUATIONS
%j
%2 IK ( I, a)
%is due to aj 2 F j well-de ned , so for a valid
%2 a UF
%8 2 N2:
%(2-8)
%The second step is to calculate IK ( I) , we calculate for each a 2 N3
%finite set U UF ( 1) UF ( 2) UF ( 3) , so that for every 2 IK ( I, a)
%Triple " = (" (1 ) " (2 ) " (3 ) ) 2 U and a 2 unit UL exist with
%( i)
%= A ( i) ' ( i)
%(1 i 3 ) :
%(2-9)
%Are fi = j ; 2 kg N3 with i < j < k and 2 k ( I, a) is arbitrarily given , then
%( i) ? ( j ) + ( j) ? (k)
%= ( I) ? (k)\nonumber\\
%so
%((I) ( j) ? ) Ij + ( ( j) ( k)? ) Jk = ik ((i) ( k)? )
%We are " ( i) : = ( i) a ( 1 i ) (1 i 3 ), it is " ( i) 2 UK
%(2-8) . We thus get the unit equation
%( i)
%(1
%i
%Gema 3)
%((I) : ( j)) aij " ij + ( ( j) : ( k)) ajk " jk = 1
%((I) : ( k)) aik " ik ((i) : ( k)) aik " ik
%over the field K , whose solution set is denoted by L . For each ('; ) 2
%L We test whether 2 UF ( 3) exists with " UF 2 (1 ) 2 UF ( 2) , and add in
%The existence of such a triple ( ' ; ; ) set U to be added .
%The third and final step in the solution of the index form equation is
%by comparing the sets U all 2 IK ( I, a) to be determined. CHECKING to
%Q
%we are for each element ( " ) 2N3 2N3 the finite set U, if a 2 oK
%can be constructed so as applies with respect to (2-9) :
%8 9 2 N3
%2 UL 8 i 2 f1 , 2, 3g :
%( i)
%= A ( i) ' (i):
%(2-10)
%So for every 2 N3 is an arbitrary but fixed gewahlter tuple " , where U 2 .
%We will assume , as a 2 IK ( I, a) exists , which (2-10) does . to
%Construction of first calculate quantities gradually
%N2 ( 1) N2 $ (2) $ $ ::: N2 () = N2
%(
%and this simultaneously for each 2 N 2) (1
%each, a " 2 UL exists with
%u = a "
%8 2 N2 ():
%Units) u 2 UL, so that
%(2-11)
%(1)
%For a 2 N3 arbitrarily we set initially N2: = f (1) , (2) , (3) g and more
%u (i ) . = " (i ) (1 i 3 ) The condition ( 2-11 ) corresponds to even ( 2-9).
%(
%For a 2 N is now a quantity N2) with known units u , and
%(
%applies without limitation N2) N2 = 6 , since otherwise we are done . There are then
%37
%(
%(
%= Fi j g 2 N2) and fj = 0 ; kg N 2 N2 N2). Be without restriction i <j <k
%There are 2 kg N3 gema (2-9 ) according to given " a unit 2 UL , to fi = j
%with
%ij =
%jk =
%ik =
%= a
%(2 ) = a
%(3 ) = a
%(1)
%(1 ) "(is satisfied o Enbar .
%remark 2.3
%(
%Was already fi ; 2 kg N2) and does not match the definition of uik in (2-12)
%the already known value for uik uberein , then ( ' ) 2N3 is not a
%2 IK ( I, a) correspond .
%With the above construction , we have for each
%determined so that
%2 N2 a unit u 2 UL
%8 2 N2
%(2-13 )
%u = a "
%applies with an unknown unit " UL 2 . By (2-3 ) and ( 2-13)
%combine , we get " out modulo torsion
%I 0 = " n (n ? 1)
%Y
%2N2
%a2 u 2 :
%Because of (2-13 ) and ( 2-2) we know as " for all fi , jg 2 N 2 , the di erences
%( i) ? ( j) , which equals Z modulo equivalence is uniquely determined. to
%based on these di erences to compute e ectively , we proceed as follows s ago. to
%! an integral basis 1 = 1 ; ::: ! ok and n of an integral basis 1 = 1 ; ::: m
%of 2 cm oL Let T n is given by
%(! 1 ; ::: n! ) = (1 ; ::: m ) T:
%Furthermore, let T1 ; ::: Tm 2 GL (m , Z ) given by
%(I (1) ; ::: i (m)) = (1 ; ::: m ) Ti
%(1 i n )\nonumber\\
%the automorphisms 1 ; ::: m 2 G are numbered such as j () = ( j) for
%all j 2 f1 ; ::: ng .
%38
%CHAPTER 2 INDEX FORM EQUATIONS
%Experience i have to put the = 1 x1 + + xn with n unknowns x1 ; ::: xn 2 Z. own
%for a fi j g 2 N2 then ( i) ? ( j) the display
%( i) ? ( j)
%Ij1 = 1 +
%with ijk Z 2 (1 k m ) , it follows
%0
%B
%B
%@
%(1 ; ::: m )
%ij 1
%..
%.
%ijm
%1
%C =
%C
%A
%+ M ijm
%( i) ? ( j)
%=
%(
%(
%(
%(
%(1i !) ; ::: ! Ni ) ) ? (! 1j) ; ::: ! Nj ) )
%=
%(
%(
%( j
%( j
%(1i ) ; ::: m)) ? (1j ) ; ::: m))
%= (( 1 ; ::: m))
%0 1
%W x 1 C. ..
%B . C
%@ A
%x
%0n 1
%W x 1 C. ..
%T B . C
%@ A
%Xn 0 1
%W x 1 .. C\nonumber\\
%(Ti ? Tj) T B . C
%@ A
%xn
%ie , we obtain for x1 ; ::: xn , the linear system of equations
%0
%B
%B
%@
%ij 1
%..
%.
%ijm
%1
%0 1
%C = (T T ? ) W x T 1 .. C:
%C i j
%B . C
%A
%@ A
%xn
%(2-14 )
%Since the di erences by ( i) ? ( j ) modulo Z- equivalence clearly defined
%, we obtain the coe cients x2 ; ::: xn
%thus , by for each
%fi , jg -free 2 N 2 , the linear system of equations (2-14 ) and subsequently end the cut -
%volume of each solution quantities Lij calculate Zn . it is
%\
%fi ; jg2N2
%Lij
%=
%0x 1011
%B x1 C B 0 C
%B 2 C + C C B :
%B .. B C. .. C
%B . C B.C
%@ A @ A
%xn
%0
%In the third step, the development of Gy } rys method is an al-
%o
%algorithm completed . Its e ciency can, however, by two simple , in
%following modi cations shown significantly increased.
%The first communication mode , which is derived from Smart 50 ] is in the second
%and third step, exploiting the action of G to N3. background is
%this fact , as for ; , " and, as in (2-9 ) is always
%((I) )
%= a
%((I) )
%( " (I)) ()
%(1 i 3 )
%39
%for every 2 G applies . Instead of the second step for each 2 N 3 , the amount
%To be determined by the toss of a unit U equation , there are only
%Sets U 1 ; ::: U t calculated . For the determination of IC ( I, a) in the third step
%U t initially
%Then we set for each ("1 ; ::: ' t) 2 U 1
%"
%i
%: = (( " I ( j 1 ) ), ( " i ( j2 ), ( " i ( j3 )))
%( I 1 t , 2 G)\nonumber\\
%wherein jk = jk (i; ) 2 f1 , 2, 3g ( i ( jk ) ) = () (k ) (1 k 3 ) , and -
%Search end followed by the method already described , whether with ( " ) 2N3
%2 IK ( I, a) vermoge (2-10) corresponds . Is this approach ? on the one hand
%the number of loose ends in the second step EinheitengleichungenQ n = j JN3
%of 3
%to t = j 3 j reduced , on the other, in the third step instead of 2N3 jU j only
%Q
%yet to examine t = 1 jU ij many tuples ( " ) 2N3 .
%i
%Basis of the second modi cation is the observation that for the successive
%(
%Construction is not necessarily the quantity N2) in the third step of the method
%every 2 N3 are needed . For any W N3 , we de ne the graph to
%: ( W ) whose vertex set is N2 and the two vertices 1 , 2 2 N2 together
%are connected if there exists W 2 = 1 second Is the graph ? ( W)
%hangend together , this indicates an inspection of the third step , since there to
%Calculation of IK ( I, a) suffices , in the second step for each 2 W, the set U
%to be determined. We illustrate this fact with a simple example .
%example 2.4
%Let n = 5 with G ' C (5). If G = h , then at a suitable numbering of the
%conjugated
%k
%i i + k mod 5
%(0 k <5 , 1 i 5 ) :
%It is then
%3
%W1 = f = f 1 = f1 , 2, 3g , f2 , 3, 4g , f3 , 4, 5g , f1 , 4, 5g , f1 , 2, 5g g\nonumber\\
%W2 = f 2 = f1 , 2, 4g , f2 , 3, 5g , f1 , 3, 4g , f2 , 4, 5g , f1 , 3, 5ggg \nonumber\\
%therefore t (5, C (5) ) = 2
%There? ( W1) hangend together , it is enough in the second step , for each of the 2 W1
%(1)
%(5)
%Set U to be calculated. We can approximately N2 ; ::: N2 as follows elections , with
%we use the terms from the iteration of page 36:
%= 1:
%= F1 , 2, 3g
%(1)
%= ) N2 = ff1 , 2g , f1 , 3g , f2 ; 3gg .
%= 2:
%= F2 , 3g , 0 = f2 , 4g =) = f2 , 3, 4g
%(2)
%= ) N2 = ff1 , 2g , f1 , 3g , f2 , 3g , f2 , 4g , f3 ; 4gg .
%= 3:
%= F3 , 4g , f3 = 0 ; 5g =) = f3 , 4, 5g
%(3)
%= ) N2 = ff1 , 2g , f1 , 3g , f2 , 3g , f2 , 4g , f3 , 4g , f3 , 5g ; f4 ; 5gg .
%= 4:
%= F4 , 5g , 0 = f1; 5g =) = f1 , 4, 5g
%(4)
%= ) N2 = ffb
%B S
%a ba
%a b
%B
%b aa S
%b
%S
%B
%b AAAS
%b B
%aa
%b
%S a
%bb
%f3 , 5g r
%B b S ARF1 ; 4g
%B bb S
%AA
%B
%S
%BRF1 ; 5g
%f3 , 4g r @ a
%a
%!
%a
%B
%!mal with the property , as appropriate for i1 ; ::: iu 2
%f1 ; ::: ? tg the graph ( Wi1 ::: Wiu ) is hangend together . Due to the first
%Modi cation , it is then sufficient , in the second step using many units u -
%equations to be solved . In the third step are accordingly i1 j jU jU j iu many
%Tuple ( " ) to investigate 2N3 .
%For all possible combinations of n 12 and jGj < Table 20 provides the
%on the next side of the overview of the values ​​of t ( n , G), and u ( n , G) , wherein the
%Name of the Galois groups of 8 ] were taken over .
%Steps 2 and 3 of our development of Gy } rys methods that we are now
%o
%in a continuation of the example 2.2 .
%Example 2.5 ( continuation of 2.2)
%According to the results already achieved remains the determination of IK ( I)
%for I 2 f5 , 7g. I = 5, so 4004 unit equations in the second step
%to loose . In most of these equations can be very quickly using the
%Criterion from 1:27 to establish because they do not have solutions. Only for a
%Unit equation need all three steps of the process from the first chapter
%be carried out . Since the solutions of these equations is not ok 2
%( ok : Z] ) = 5 can be constructed IK applies (5) = , . The result for this
%required computation time was 743S total . Of this, 139 seconds on
%the second step , that is the equation of the unit lots , less than
%41
%n
%3
%4
%5
%6
%7
%8
%9
%10
%11
%12
%G
%A3
%S3
%C (4)
%E ( 4)
%D (4)
%C (5)
%D (5 )
%C (6)
%D6 (6 )
%D ( 6)
%A4 ( 6)
%F18 (6 )
%C (7)
%D ( 7)
%C (8)
%jGj t (n , G ) u (n , G )
%3
%6
%4
%4
%8
%5
%10
%6
%6
%12
%12
%18
%7
%14
%8
%4] 2
%8
%E ( 8)
%8
%D8 ( 8)
%8
%Q8 ( 8)
%8
%D ( 8)
%16
%1 23 ] 4
%16
%2
%2D8 (8)
%16
%E (8): 2
%16
%24 ] 2
%16
%1 23 ] E ( 4)
%16
%2
%C (9 )
%9
%E ( 9)
%9
%D ( 9)
%18
%S ( 3)] 3
%18
%1
%S ( 3) 2] S ( 3) 18
%C (10)
%10
%D (10 )
%10
%C (11)
%11
%C (4) ] C (3) 12
%E ( 4)] C (3) 12
%D6 (6)] 2 12
%A4 ( 12)
%12
%1 3: 2] 4
%12
%2
%1
%1
%1
%1
%1
%2
%2
%4
%4
%3
%4
%2
%5
%4
%7
%7
%7
%7
%7
%5
%5
%5
%5
%5
%5
%10
%12
%7
%8
%8
%12
%12
%15
%19
%19
%19
%21
%19
%1
%1
%1
%1
%1
%1
%1
%2
%2
%1
%2
%1
%2
%1
%2
%3
%3
%3
%3
%2
%2
%2
%2
%2
%2
%2
%2
%2
%3
%4
%4
%4
%4
%5
%5
%5
%6
%5
%42
%CHAPTER 2 INDEX FORM EQUATIONS
%Second to the third step and the big e residue on the first step , ie the
%Determination of A.
%It now is known already iK = 7th As in the case I = 5 may 5005 for part of the Great
%Unit equations to consider the explicit calculation of IK ( 7)
%are easily determined because they have no solutions. From the remaining
%ten unit equations which solved by the methods of the first chapter
%must be , for IK ( 7) results in the following 25 -element Vetretersystem :
%IK (7) = f 4 + 5 ! \nonumber\\
%4! 3 ? 3 ! 4 ? 4! 5; ! 2; ! 2 + ! 3 ? ! 4 ? ! 5\nonumber\\
%2 ! 2 ? 3 ! 3 + 2 ! 4 + 3 ! 5; 2 ! 2 + ! 3 ? ! 4 ? ! 5 , 2 ! 2 + 6! 3 ? 5 ! 4 ? 6! 5\nonumber\\
%3 ! 2 ? 3 ! 3 + ! 4 + 2 ! 5 , 3 ! 2 ? 3 ! 3 + 2 ! 4 + 3 ! 5 , 3 ! 2 ? 2 ! 3 + ! 4 + 2 ! 5\nonumber\\
%4! 2 ? 3 ! 3 + ! 4 + 2 ! 5 , 4 ​​! 2 ? 2 ! 3 + ! 4 + 2 ! 5 , 5 ! 2 ? 2 ! 3 + 5 ? \nonumber\\
%5 ! 2 ? 2 ! 3 + ! 4 + 2 ! 5, 6 ! 2 ? 6! 3 + 3 ! 4 & 5 ! 5, 6 ! 2 ? 2 ! 3 + 5 ? \nonumber\\
%7! 2 ? 6! 3 + 3 ! 4 & 5 ! 5 , 8! 2 ? 5 ! 3 + 2 ! 4 + 4! 5\nonumber\\
%9! 2 ? ! 3 5 2 4 + 4 + 5; ! 10 2 ? 9 3 + 4 4 + 5 7 ! \nonumber\\
%13! 2 ? ! 7 3 3 4 + 5 + 6 ; ! 14 2 ? 12 3 + 7 + 4 11 5 ! \nonumber\\
%15! 2 ? ! 8 3 3 4 + 5 + 6 ; ! 22 2 ? 13 3 4 4 + 5 + 9 ! \nonumber\\
%23! 2 ? ! 11 3 + 3 + 4 8 5 g :
%The total computation time was 1442s with 779s for the second step and about
%a second for the third step.
%An inspection of the result shows in other respects , since oK exactly two classes of non-
%isomorphic equation systems of index 7 has . The orders of the first
%Have Z- equivalence class modulo 2 power base generator , while the Ordnun -
%tions of the second class each about 3 power base generator modulo Z- equivalence
%grout .
%Besides this example, we have rys with Gy } method index form equations in
%o
%Kreisteilungskorpern and its maximum real Teilkorpern solved . in fact
%we determined where in each case all power wholeness bases modulo Z- equivalence , ie
%more precisely , we calculated IKM ( 1) and IK + ( 1 ) for all m 2 N , m 6 2 mod 4 , with
%m
%Km : Q] 12, wherein for the eingefuhrte in subsection 1.3.2 Notation
%Kreisteilungskorper be recalled . The results of these calculations are in Ta -
%belle arrested on page 43rd
%For Kreisteilungskorper higher degree is the use of Gy } rys method very
%o
%consuming ( see, eg, Q ( 13) in the table) . The cause hierfur is both in the
%second step , ie the unit lots of equations , as well as the com-
%combination of u (n , G ) many sets U in the third step , the cardinality
%these quantities with increasing unit rank erfahrungsgema increases greatly .
%For special Kreisteilungskorper , namely when m is a prime number , can
%used , however, of a Niklasch 39, Section II -4.3 ] developed process
%be that based on the full power of all units except bases of Km mo -
%modulo Z- equivalence determined. The main advantage of this method consists in that
%43
%m
%1
%3
%4
%5
%7
%8
%9
%11
%12
%13
%15
%16
%20
%21
%24
%28
%36
%+
%Km : Q] jik + (1) j
%m
%1
%1
%1
%2
%3
%2
%3
%5
%2
%6
%4
%4
%4
%6
%4
%6
%6
%t
%Km : Q] jIKm ( 1) j
%1
%{
%1
%{
%1
%{
%1
%{
%9
%3s
%1
%{
%6
%2s
%25
%47s
%1
%{
%36 2576s
%12
%27s
%6
%24s
%10
%23s
%30 1750s
%6
%27s
%15 639s
%15 681S
%1
%2
%2
%4
%6
%4
%6
%10
%4
%12
%8
%8
%8
%12
%8
%12
%12
%1
%1
%1
%6
%9
%2
%9
%15
%4
%18
%16
%4
%8
%24
%8
%12
%12
%t
%{
%{
%{
%0s
%15s
%0s
%50s
%2900s
%0s
%34195s
%891s
%303s
%951s
%32872s
%804S
%31004s
%21066s
%its cost is linear in the number of exception units. We are this
%Procedures now represent short and wholeness so that all power bases of Q ( 17) ,
%Q (19) and Q ( 23) directions .
%For any prime p > 3 p is a primitive p- th root of unity . the
%(
%j
%Conjugate of p are numbered as pj ) = p ( y 1 < p) .
%(
%We put $: = 1 ? p . For each k 2 f1 ; ::: p ? 1g is 1 ? pk ) associated to $
%k
%because N (1 ? p) = N (1 ? p) and
%k
%1 ? p
%1 ? p = 1 p + +
%k
%+ P 1 2 OKP ? :
%(
%(
%So for i , j 2 f1 ; ::: p ? 1g, i 6 = j , and any pi) ? pj ) associated to $ due
%( i)
%( j)
%i
%j ? i
%p ? p = p (1 p ? )
%Let 2 IKP ( 1) arbitrary but fixed chosen . Gema (2-2 ) and ( 2-4) (i ) ? ( j) for
%(
%(
%i , j 2 f1 ; ::: p ? 1g, i 6 = j , associated to pi) ? pj ) , and thus also associated to $ .
%For each k 2 Z , p 6 j k , is thus
%! k: = (1 ?)
%?
%(k)
%(1)
%(1)
%( 2-15)
%a unit in OKP , the conjugate should be read in (2-15 ) modulo p | a
%Convention that we maintain the following . For j , k 2 Z , p 6 j jk , then
%!
%(
%!1). There! G of (2-17 ) is an exceptional unit can for all possibilities ! G
%will go through the calculation of XKP . We will now see as
%modulo Z- equivalence g can be reconstructed only by specifying ! . to
%We determine first ? g using (2-17 ) and then using ( 2-16) gradually
%(
%! ( ? ? g g ) ! g2 = gg ) + g ! \nonumber\\
%(
%! ( ? ? g g ) ! g3 = gg ) + g ! \nonumber\\
%2
%..
%..
%.
%.
%(
%? ( ? ? g g ) ! gp 2 = gg ) 3 + g ? :
%p
%p
%We come in as p = a2 + + ap 1 p 2 with unknowns a2 ; ? ::: ; ? Ap 1 2 Z.
%( ? 1) ? ( 1) = "$ with an unknown unit " 2 o . place
%It should also be
%kp
%we
%1
%0
%(p ? 2) 2
%B
%A: B =
%@
%p ? p
%2
%p ? p
%4
%2 (p ? 1)
%..
%.
%p ? 1
%p ? p
%?
%:::
%2
%p
%2
%p
%then A is regular because det2 A = discKp . from
%p
%..
%; ::: ap 1 2 Z are uniquely determined up to sign ? .
%Based on the results achieved in the last chapter except to units in district -
%teilungskorpern we have with this method, all the power of wholeness bases in Q ( 17) ,
%Q (19) and Q (23 ) is determined . The results of these calculations and the Ta -
%Table of page 43 of a match Bremner 7] GEAU Erten presumption that
%so far only for p 7 was veri ed :
%Assumption 2.6 ( Bremner )
%(
%0
%0
%Let p 2 OKP de ned by p : = p + pp + 1) = 2 ? . Then there exists for every
%2 IKP ( 1) is an automorphism of the Galois group of Kp = Q , so that
%0
%either Z- is equivalent to (p) or (p).

\nocite{*} % Insert publications even if they are not cited
\bibliographystyle{alpha}
\bibliography{biblio}

\end{document}
