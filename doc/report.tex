\author{Alex J. Best \\Supervised by Dr. Lassina Demb\'el\'e}
\date{\today}
\title{Finding orders with prescribed index in number fields}
\documentclass[11pt,a4paper]{report}
\usepackage{amsmath, amssymb, amsfonts, amsthm, hyperref}
\usepackage[utf8]{inputenc}
\usepackage[T1]{fontenc}
\usepackage[english]{babel}

\newtheorem*{thm}{Theorem}
\newtheorem*{lem}{Lemma}
\newtheorem*{cor}{Corollary}
\newtheorem*{prop}{Proposition}
\theoremstyle{definition}
\newtheorem*{defn}{Definition}
\newtheorem*{defns}{Definitions}
\newtheorem*{ex}{Example}
\newtheorem*{rem}{Remark}
\newtheorem*{nota}{Notation}
\newtheorem*{alg}{Algorithm}

\setcounter{tocdepth}{3}

\begin{document}
\maketitle
\tableofcontents

% TODO list:
% - Add sage and magma to refs.
% - Include thanks?

\chapter{Introduction}

\chapter{Background material}
In this section we fix several definitions and important results from algebraic number theory and commutative algebra.
These results are well known and are used throughout the rest of the report.



\chapter{Solving the problem}


\chapter{Applications}


\chapter{Conclusion}


\chapter{Appendix: Code}

Much of what was done has been implemented in both the Sage and Magma and so we provide annotated source code listings for the algorithms in both languages below.
 
\section{Sage}


\section{Magma}


\nocite{*}
\bibliographystyle{alpha}
\bibliography{biblio}



\end{document}
